\documentclass{article}

\usepackage[utf8]{inputenc}

\author{Rémy Besognet \& Pierrick Couderc}

\date{24 mars 2014}

\title{Projet SMT}

\begin{document}

\maketitle

\section{Utilisation}

Tout d'abord, la compilation se fait via la commande \texttt{make}, et peut se
faire soit en bytecode, soit en code natif (\texttt{make opt}).

L'exécutable ainsi créé est \texttt{so-many-theories}, dont l'utilisation est la
suivante (les options seront détaillées plus loin) :
\begin{center}
  \texttt{./so-many-theories[.opt] <options> <file>.(cnf|cnfuf)}
\end{center}

L'exéctuable peut prendre deux types de fichiers, tous deux au format
\emph{quasi}-DIMACS du projet :
\begin{itemize}
\item \textbf{.cnfuf} : format des fichiers standard.
\item \textbf{.cnf} : même format que le .cnfuf, mais avec directement des
  variables à la place des équations. Utiles pour tester la partie \emph{SAT}.
\end{itemize}

\section{Implémentation}


Pour la partie \emph{SAT} du solver, DPLL et CDCL ont tous deux été
implémentés. En revanche, CDCL étant plus efficace c'est celui qui est retenu
comme algorithme par défaut. La version de DPLL est la plus simple, sans
backjump donc. Elle est toujours accessible via l'option \texttt{-dpll}.

Pour CDCL, il y a donc bien le backjump et l'apprentissage, mais également VSIDS
pour le choix des variables de décision. La période et la constante pour diviser
les scores sont d'ailleurs paramétrables via les options \texttt{-period} et
\texttt{div}. Enfin, VSIDS est totalement désactivable via \texttt{-no-vsids},
en particulier pour tester l'impact de cette heuristique (qui est largement
visible).

La partie intéressante est que le SAT-solver est functorisé, et prend en
paramètre un module \emph{TheorySolver} qui symbolise la partie théorie de
SMT. Du coup, la théorie des booléens résolue par SAT n'est rien qu'un module
qui ne fait rien d'autre que de suivre l'avis de SAT et ne fait rien de ses
arguments pour chacune de ses fonctions. C'est ce qui permet de tester que les
algorithmes DPLL et CDCL fonctionnent correctement. L'égalité est donc encodée
dans un tel module, et \emph{en théorie} il devrait être possible de brancher
n'importe quel solveur pour une théorie (ou une combinaison) et d'utiliser la
partie SAT du projet.

Ce type de solveur a une fonction principale qui est \texttt{add\_literal}, qui
ajoute un litéral à la théorie et retourne une nouvelle structure de donnée qui
symbolise cette théorie si cet ajout est correct, ou \texttt{None} si l'ajout
est contraire à la théorie. L'appel à cet fonction est fait à chaque ajout d'un
literal (dans les règles Unit et Decide), ce qui permet de
backtracker/backjumper dès que nécessaire.

\medskip

Explication sur égalité ?


\end{document}
